\documentclass[a4paper]{article}

\usepackage[utf8]{inputenc}
\usepackage[T1]{fontenc}
\usepackage[french]{babel}
\usepackage[top=1cm, left=1cm, bottom=1cm, right=1cm]{geometry}
\usepackage{setspace}
\usepackage{soul}
\usepackage{ulem}
\usepackage{color}
\usepackage{xcolor}
\usepackage{listings}
\usepackage{upquote}
\usepackage{url}
\usepackage{graphicx}
\usepackage{wrapfig}
\usepackage{float}
\usepackage{multirow}
\usepackage{array}
\usepackage{colortbl}
\usepackage{amsmath}
\usepackage{amssymb}
\usepackage{mathrsfs}
\usepackage{mathtools}
\usepackage{amsthm}
\usepackage{makeidx}
\usepackage{empheq}
\usepackage{pgf,tikz}
\usepackage{xargs}
\usepackage{subcaption}
\usepackage{tabularx}
\usepackage{hhline}
\usepackage{enumitem}
\usepackage{fancyhdr}
\usepackage{titling}
\usepackage{dashbox}
\usepackage{eqparbox}
\usetikzlibrary{arrows, arrows.meta, bending, calc, backgrounds, patterns}
\usetikzlibrary{decorations.markings}

\newcommand\alignsymbols[2][c]{\mathrel{\eqmakebox[S][#1]{$#2$}}{}}
\input{../templates/styles/gfsartemisia.tex}

\author{BENICHOU Nathan}
\title{Exercice 5 du troisième TD d'intégration}

\renewcommand{\theenumi}{\textbf{\arabic{enumi}.}}
\renewcommand{\theenumii}{\textbf{\alph{enumii}.}}
\renewcommand{\theenumiii}{\textbf{\roman{enumiii}.}}

\renewcommand{\labelenumi}{\theenumi}
\renewcommand{\labelenumii}{\theenumii}
\renewcommand{\labelenumiii}{\theenumiii}
\geometry{top=2.25cm, left=1cm, bottom=2.25cm, right=1cm}

\pagestyle{fancy}
\renewcommand{\footrulewidth}{0pt}
\fancyhf{}
\lhead{\textsc{\theauthor}}
\rhead{\thetitle}
\definecolor{alizarin}{rgb}{0.905882353,0.298039216,0.235294118}
\definecolor{nephritis}{rgb}{0.152941176,0.682352941,0.376470588}
\definecolor{belizehole}{rgb}{0.160784314,0.501960784,0.725490196}
\definecolor{peterriver}{rgb}{0.203921569,0.596078431,0.858823529}

\definecolor{officeblueint}{rgb}{0.51372549,0.745098039,0.925490196}
\definecolor{officeblueext}{rgb}{0,0.388235294,0.694117647}
\definecolor{officegreenint}{rgb}{0.631372549,0.866666667,0.666666667}
\definecolor{officegreenext}{rgb}{0.188235294,0.564705882,0.282352941}
\definecolor{officeredint}{rgb}{1,0.568627451,0.596078431}
\definecolor{officeredext}{rgb}{0.831372549,0.137254902,0.078431373}
\definecolor{officeyellowint}{rgb}{0.97254902,0.858823529,0.560784314}
\definecolor{officeyellowext}{rgb}{0.870588235,0.423529412,0}
\definecolor{officepurpleint}{rgb}{0.831372549,0.57254902,0.847058824}
\definecolor{officepurpleext}{rgb}{0.62745098,0.294117647,0.658823529}
\definecolor{officegrayint}{rgb}{0.784313725,0.776470588,0.768627451}
\definecolor{officegrayext}{rgb}{0.474509804,0.466666667,0.454901961}
\definecolor{officeblackint}{rgb}{0.784313725,0.776470588,0.768627451}
\definecolor{officeblackext}{rgb}{0.22745098,0.22745098,0.219607843}
\newcommand\dashedph[1][H]{\setlength{\fboxsep}{0pt}\setlength{\dashlength}{2.2pt}\setlength{\dashdash}{1.1pt} \dbox{\phantom{#1}}}

\newcommand\indep{\perp \!\!\! \perp}
\newcommand\Esp[1]{\mathbb{E}(#1)}
\newcommand\Prob[1]{\mathbb{P}(#1)}
\newcommand\Ber[1]{\mathcal{B}(#1)}
\newcommand\N{\mathbb{N}}
\newcommand\Z{\mathbb{Z}}
\newcommand\R{\mathbb{R}}
\newcommand\Q{\mathbb{Q}}
\newcommand\M{\mathscr{M}}
\newcommand\norm[1]{\left\lVert#1\right\rVert}

\newtheorem*{prop}{Propriété}
\newtheorem*{propn}{Proposition}
\newtheorem*{lem}{Lemme}
\newtheorem*{crl}{Corollaire}
\newtheorem*{thm}{Théorème}
\newtheorem*{defn}{Définition}

\begin{document} %------------------------------------------------------------------------------------------------------------------------------------------------- DOC
\section{Questions}
\noindent Pour $n\geqslant 1$, on définit :
$$f_n:\begin{aligned}
    \R_+ \longrightarrow & \ \R_+^* \\
    x    \longmapsto     & \ \dfrac{n^x}{\displaystyle\prod_{1\leqslant k\leqslant n}\left(1+\dfrac{x}{k}\right)} = \dfrac{n^xn!}{\displaystyle\prod_{1\leqslant k\leqslant n}\left(x+k\right)}
\end{aligned}$$
\begin{enumerate}
    \item Pour $n\geqslant 1$, on a : $f_n > 0$ donc on peut calculer son logarithme : pour $x\geqslant 0$,
    \begin{align*}
        \ln f_n(x)&\alignsymbols{=} x\ln n - \displaystyle\sum_{1\leqslant k\leqslant n}\ln\left(1+\dfrac{x}{k}\right) \\
        &\alignsymbols{=} x\displaystyle\sum_{1\leqslant k < n}\left(\ln\left(k+1\right)-\ln k\right) - \displaystyle\sum_{1\leqslant k<n}\ln\left(1+\dfrac{x}{k}\right) - \ln\left(1+\dfrac{x}{n}\right) \\
        &\alignsymbols{=} \displaystyle\sum_{1\leqslant k<n}\left(x\ln\left(1+\dfrac{1}{k}\right)-\ln\left(1+\dfrac{x}{k}\right)\right)-\ln\left(1+\dfrac{x}{n}\right)\\
        &\alignsymbols{\underset{n\rightarrow +\infty}{=}} \displaystyle\sum_{1\leqslant k<n}\left(x\dfrac{1}{k}-\dfrac{x}{k}+O\left(\dfrac{1}{n^2}\right)\right)+o(1)
        \alignsymbols{\underset{n\rightarrow +\infty}{=}} \displaystyle\sum_{1\leqslant k<n}O\left(\dfrac{1}{n^2}\right)+o(1)
    \end{align*}
    On constate que $(\ln f_n)$ converge simplement, donc $(f_n)$ converge simplement vers une fonction positive $f:\R_+\longmapsto\R_+$.
    \item Soient $n\geqslant 1$ et $x\geqslant 0$,
    \begin{align*}
        f_n(x+1)&=\dfrac{n^{x+1}n!}{\displaystyle\prod_{1\leqslant k\leqslant n}\left(x+1+k\right)}\\
        &=n\dfrac{n^xn!}{\displaystyle\prod_{2\leqslant k\leqslant n+1}\left(x+k\right)}\\
        &=n\dfrac{n^xn!}{\dfrac{1}{x+1}(x+n+1)\displaystyle\prod_{1\leqslant k\leqslant n}\left(x+k\right)}\\
        &=\dfrac{n}{n+x+1}(x+1)\dfrac{n^xn!}{\displaystyle\prod_{1\leqslant k\leqslant n}\left(x+k\right)}\underset{n\rightarrow+\infty}{\sim}(x+1)f_n(x)\\
    \end{align*}
    D'où : $f(x+1)=(x+1)f(x)$.
    \item Pour $n\geqslant 1$, $f_n\in\mathcal{C}^\infty(\R_+)$ et $f_n>0$, alors on note : $$g_n=\dfrac{f_n'}{f_n}=(\ln f_n)'$$
    Grâce à l'expression de $\ln f_n$ trouvée en (1.), on a directement, pour $x\geqslant 0$ :
    $$g_n(x)=\ln n - \displaystyle\sum_{1\leqslant k\leqslant n}\dfrac{1}{x+k}\text{\ \ \ \ \ \ \ \ \ et donc :\ \ \ \ \ \ \ \ \ }g_n'(x)=\sum_{1\leqslant k\leqslant n}\dfrac{1}{(x+k)^2}$$
    Par un calcul similaire à celui de (1.), on a : $$g_n(x)=\displaystyle\sum_{1\leqslant k<n}\left(\ln\left(1+\dfrac{1}{k}\right)-\dfrac{1}{x+k}\right)-\dfrac{1}{x+n}\underset{n\rightarrow +\infty}{=}
    \displaystyle\sum_{1\leqslant k<n}\left(\dfrac{1}{k}-\dfrac{1}{x+k}+O\left(\dfrac{1}{n^2}\right)\right)+o(1)\underset{n\rightarrow +\infty}{=}\displaystyle\sum_{1\leqslant k<n}O\left(\dfrac{1}{n^2}\right)+o(1)$$
    Donc $(g_n)$ converge simplement vers une fonction $g:\R_+\longmapsto\R$. De plus, comme $\dfrac{1}{(x+n)^2}\underset{n\rightarrow +\infty}{=}O\left(\dfrac{1}{n^2}\right)$,
    on a aussi que $(g_n')$ converge simplement vers une fonction $h:\R_+\longmapsto\R$. Appliquons maintenant le lemme 1 sur $g_n'$ :
    $$g_{n+1}'(x)-g_n'(x)=\dfrac{1}{(x+n+1)^2}$$
    On constate que $g_{n+1}'-g_n'$ est décroissante et positive sur $\R_+$, on conclut que $(g_n')$ converge uniformément vers $h$ sur $\R_+$. \\
    Ainsi, d'après le lemme 3, $(g_n)$ converge uniformément vers $g$ sur tout segment $I\subset\R_+$, $g$ est dérivable sur $\R_+$ et $g'=h$. \\
    On applique encore le lemme 3 et on obtient : $(\ln f_n)$ converge uniformément vers $\ln f$ sur tout segment $I\subset\R_+$, $\ln f$ est dérivable sur $\R_+$ et $(\ln f)'=g$.
    Ainsi $f$ est dérivable sur $\R_+$ et $f'=fg$. Avec ces résultats, on peut dire que $f\in\mathcal{C}^2(\R_+)$.\\
    D'ailleurs, grâce au lemme 2, on peut aussi conclure que $f_n$ converge uniformément vers $f$ sur tout segment $I\in\R_+$.
\end{enumerate}
\section{Lemmes}
\end{document}