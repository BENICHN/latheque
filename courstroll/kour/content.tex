\chapter{Suprématie plotiste}

\begin{res}
  L'expérience nous montre que ce chapitre-là est injustement négligé par la plupart des professeurs de France.
  Pourtant, il reste un domaine des mathématiques au programme. Aujourd'hui, nous n'allons pas nous priver de l'explorer de long en large.

  Dans un premier temps, nous nous chargerons de poser les bases de cette algèbre. Puis nous nous intéresserons à l'utilité-même de ce chapitre :
  démontrer la suprématie plotiste (ce qui passe par la suprématie relative à chaque filière puis la suprématie générale).
\end{res}

\minitoc

\begin{att}{Disclaimer}
  Pour pouvoir lire et comprendre ce cours, il est nécessaire (et suffisant) de poser son cerveau sur sa table.
\end{att}

\clearpage

\section{Problèmes}

\setcounter{numeroprob}{158}

\begin{probleme}{Mettre au point l'algèbre fermatienne}
  Comment, au sein de notre magnifique institution des Classes Préparatoires aux Grandes Écoles du Lycée Pierre de Fermat de Toulouse,
  poser des règles d'interaction des classes les unes avec les autres ?
  Comment gérer leurs structures d'ensemble et leurs lois de composition internes ? Autrement dit, quelles portes nous ouvre l'algèbre fermatienne ?
\end{probleme}

\begin{voir}{Suprématie plotiste}
  \includegraphics[width=\textwidth]{images/podium.png}
\end{voir}


\begin{probleme}{Comment classer les classes les unes par rapport aux autres ?}
  Quand on prend des exemples particuliers dans le lycée comme les ECS ou les MP2, force est de constater que nous leur sommes supérieurs.
  Est-il possible d'établir une relation d'ordre (partielle ou totale) entre les différentes classes de Fermat ?
  Dans ce cas-là, pouvons-nous nous intéresser plus globalement à la position de la classe de MPSI3 par rapport aux autres classes,
  quelles qu'elles soient, c'est-à-dire toutes les fusions, intersections, multiplications (!) entre différentes classes ?
\end{probleme}

\section{Généralités}

\newcommand\proj{p}
\newcommand\Prob{\mathbb{P}}
\newcommand\Part{\mathcal{P}}
\newcommand\EF{\mathbb{E}_{\tn{Fer}}}
\newcommand\PF{\mathbb{P}_{\tn{Fer}}}
\newcommand\AF{\mathbb{A}_{\tn{Fer}}}
\newcommand\FF{\mathbb{F}\tn{ermat}}
\newcommand\Car{\mathbb{C}\tn{ar}}
\newcommand\Fil{\mathbb{F}\tn{il}}
\newcommand\tutu{$-\hspace{0.4pt}-$}
\newcommand\tn[1]{\textnormal{#1}}

\subsection{Espace vectoriel des classes dans Fermat}

\begin{defn}{$\EF$ et $\Car$}
  On appelle $\EF$ l'ensemble des élèves étudiant à la prépa du lycée Fermat.
  On notera $\Car$ l'ensemble des caractéristiques que peut avoir un étudiant de $\EF$.
  Un élément de $\Part(\EF)$ est appelé classe et un élément de $\Car$ est appelé caractère.
  Par convention, les classes sont écrites en capitales et les caractères en minuscules.

  On peut additionner les éléments de $\Car$ et de $\Part(\EF)$ (respectivement entre eux)
  selon la loi $\cup$ (d'inverse $\cap$) et multiplier les éléments de $\Car$ par les éléments de $\Part(\EF)$ ou de $\Car$ selon la loi $\times$.
\end{defn}

\begin{ex}{Pour bien comprendre}
  Quelques classes et caractéristiques :
  \begin{itemize}
    \item La MPSI3 est la classe constituée des élèves de $\EF$ :\\
          \{AncelinLisa, AvenatSylvain, \dots, WindelsArthur\}
    \item $\tn{talentueux}\times\tn{MPSI2}= \tn{MPSI3}$
    \item $\tn{MPSI} = \tn{MPSI1} \uplus \tn{MPSI2} \uplus \tn{MPSI3}$
    \item $\tn{3hPC}\ \cup\ \tn{2hMaths} = \tn{SiesteMardiaprem}$
    \item $\tn{samedi} \times \tn{RéfNonGaliléen} = \tn{Décès}$ \\
  \end{itemize}
\end{ex}

\begin{rem}{Nomenclature}
  La plupart des éléments de $\EF$ et des multiplications de caractères par des classes ne correspondent à aucun élément connu.
  Par exemple, la classe de tous les élèves de $\EF$ dont le nom fait sept lettres ne porte aucun nom. \\
  De plus, $\tn{sérieux} \times \tn{MPSI3} = \emptyset$

  On s'accordera pour appeler filière les classes portant un nom (comme MPSI3…) et $\Fil$ l'ensemble des filières.

  Les élèves sont considérés indifféremment selon s'ils sont spés ou sups. En effet, s'ils sont spés à Fermat,
  c'est qu'ils ont déjà travaillé en tant que sup.
  On choisit donc de les associer à la classe dans laquelle ils étaient lorsqu'ils étaient en sup.
  Ainsi, $\tn{Élie} \in\tn{MPSI3}$\\

  Cependant $\EF$ est mutable. En effet, lorsque les élèves quittent Fermat, ils sont supprimés de l'ensemble. De même, lorsqu'ils intègrent Fermat, les élèves sont
  ajoutés à l'ensemble.
\end{rem}\ \\

\begin{rem}{Bruni}
  $\tn{MPSI3}\ \cap\ \tn{HK} = \tn{Bruni}$ mais Bruni est un élément négligeable. \\
  Ainsi $\tn{MPSI3}\ \cap\ \tn{HK} \sim 0$
\end{rem}

\begin{plusloin}{Ensemble $\FF$}
  D'autres éléments comme $\PF$ (l'ensemble des professeurs) ou
  $\AF$ (l'ensemble de l'administration, des agents d'entretien et cuisiniers) existent.

  On a de plus : $\FF = \AF \uplus \EF \uplus \PF$
\end{plusloin}

\begin{prop}{Structure de $\Part(\EF)$}
  $(\Part(\EF), \cup, \times)$ est un $\Car$-espace vectoriel.
\end{prop}

\begin{exercice}
  Faire la démonstration.
\end{exercice}

\begin{correction}
  Trivialissime : il suffit d'utiliser la définition.\\
  Soient $c1, c2\in \Car$ et $C\subset \EF$ :
  \begin{align*}
    c1\times(c2\times C) & =c1\times\{c2\times e | e \in C\}                          \\
                         & =\{c1\times (c2\times e) | e \in C\}                       \\
                         & =\{(c1\times c2)\times e | e \in C\}=(c1\times c2)\times C
  \end{align*}
  Soient $c1, c2\in \Car$ et $C1, C2\subset \EF$ :
  \begin{align*}
      & (c1\cup c2)\times (C1\cup C2)                                                                                        \\
    = & \{(c1\cup c2)\times e\ |\ e\in C1 \cup C2\}                                                                          \\
    = & \{c1\times e\ |\ e\in C1\}\cup\{c1\times e\ |\ e\in C2\}\cup\{c2\times e\ |\ e\in C1\}\cup\{c2\times e\ |\ e\in C2\} \\
    = & c1\times C1\cup c1\times C2\cup c2\times C1\cup c2\times C2
  \end{align*}
  Soit $C\subset \EF$ :
  \begin{align*}
    \tn{id}\times C = \{id\times e\ |\ e\in C\} = \{e\ |\ e\in C\} = C
  \end{align*}
  Ainsi $(\Part(\EF), \cup, \times)$ est un $\Car$-e.v..
\end{correction}

\begin{defn}{Module et relation ($\equiv$)}
  Pour une classe donnée $C$, on utilisera la notation $\proj_C : \Part(\EF) \rightarrow \Car$
  pour désigner le caractère de la composante sur $C$ des éléments de $\Part(\EF)$.

  On notera aussi ($\equiv$) la relation telle que : \\
  $\forall C1, C2 \in \Part(\EF),\ C1 \equiv C2 \Leftrightarrow \proj_{C1} =  \proj_{C2}$
\end{defn}

\begin{prop}{Nature de la relation ($\equiv$)}
  La relation ($\equiv$) est une relation d'équivalence.
\end{prop}

\begin{dem}
  Ses classes d'équivalence sont les images réciproques de chaque projection.
\end{dem}

\begin{rem}{Inclusion}
  On a toujours la relation d'ordre ($\subset$) entre les classes.
\end{rem}

\begin{att}{Ordre partiel}
  La relation d'inclusion n'est pas totale.
  Prenons Jonathan, Baptiste, Alexis $\in$ Coinche :
  \{Jonathan, Baptiste\} et \{Baptiste, Alexis\} ne sont pas comparables.
\end{att}

\begin{defn}{Sous-classe}
  Une partie d'une classe $C$ est appelée sous-classe de $C$.
\end{defn}

\begin{prop}{Sous-espace vectoriel}
  Pour toute classe $C$ dans $\Part(\EF)$, l'ensemble des sous-classes de $C$
  est un s.e.v. de $\Part(\EF)$.
\end{prop}

\begin{dem}
  Il suffit de l'écrire.
\end{dem}

\begin{prop}{Somme directe}
  $\EF =  \tn{HK} \uplus \tn{Chartes} \uplus \tn{Bio} \uplus \tn{PCSI} \uplus \tn{MPSI}$

  Ainsi : $\Part(\EF) = \Part\tn{(HK)} \oplus \Part\tn{(Chartes)} \oplus \Part\tn{(Bio)} \oplus \Part\tn{(PCSI)} \oplus \Part\tn{(MPSI)}$ \\
  où : $\Part\tn{(Bio)} = \Part\tn{(Grenouilles)} \oplus \Part\tn{(Poireaux)}$,\\
  $\Part\tn{(PCSI)} = \Part\tn{(PCSI1)} \oplus \Part\tn{(PCSI2)}$ et\\
  $\Part\tn{(MPSI)} = \Part\tn{(MPSI1)} \oplus \Part\tn{(MPSI2)} \oplus \Part\tn{(MPSI3)}$
\end{prop}

\subsection{Base de l'espace des classes}

\begin{prop}{Dimension finie (Théorème de l'avantage concurrentiel)}
  Pour toute filière $F$ de $\Fil$, l'espace $\Part(F)$ possède une base canonique, également appelée salle de base.
\end{prop}

\begin{voir}{Beckbingo}
  \includegraphics[width=\textwidth]{images/beck.jpeg}
\end{voir}

\begin{dem}
  Dans le discours de Beckrich du début de l'année, le proviseur a dit lui-même la phrase suivante :
  “Nous n'avons pas les moyens d'ouvrir les salles” (cf Beckbingo à gauche) pour faire référence à
  l'impossibilité financière d'administrer l'entièreté des salles du lycée.
  Cependant, il se fonde sur un principe de mutuelle confiance entre l'administration et les élèves pour donner
  un libre accès aux salles après les cours. Ainsi, pour toute la soirée, les classes disposent d'une salle de base.
  Donc pour toute filière $F$ de $\Fil$, il existe une base $\mathcal{B}$ de $\Part(F)$.
\end{dem}

\subsection{Palindromes}
Pour démontrer la supériorité, il va falloir rajouter des outils
extérieurs. Ici nous allons parler de la fonction pal.

\begin{defn}{Lettre, mot, Mot, pal et palindrome}
  Soit $n \geqslant 0$. On appelle lettre un élément de l'alphabet ou le caractère vide “\_”,
  mot un $n$-uplet de lettres et Mots l'ensemble des mots.
  On introduit la fonction :
  $$\tn{pal} : \ \begin{aligned}
      (\tn{Alphabet}\cup\{“\_”\})^n             & \rightarrow(\tn{Alphabet}\cup\{“\_”\})^n             \\
      (\tn{lettre}_i)_{1\leqslant i\leqslant n} & \mapsto(\tn{lettre}_{n-i})_{1\leqslant i\leqslant n}
    \end{aligned}$$
  On appelle palindromes les mots de deux lettres ou plus invariants par pal.

  L'ensemble des palindromes est appelé Palindromes.

  Sur Mots sont définies deux lois internes :
  \begin{itemize}
    \item $(++)$ est la concaténation de deux mots.
    \item $(+)$ est la superposition de deux mots lorsque deux caractères non vides ne se chevauchent pas.
  \end{itemize}
\end{defn}

\begin{plusloin}{verlan}
  Une autre fonction de Leturcq\tutu Daligaux est \textnormal{verlan}.
  Historiquement parlant, le débat a été long pour déterminer s'il s'agissait d'une symétrie.
  Finalement, le célèbre mathématicien a apporté l'argument décisif de l'arabe et a ainsi clôt le débat.
  En effet, le verlan de “Arabe” est “Rebeu” et le verlan de “Rebeu” est “Beur” or “Arabe” $\ne$ “Beur”
  donc verlan n'est pas une symétrie.
\end{plusloin}

\begin{ex}{Palindromes}
  Kayak, Bob, Anna $\in$ Palindromes \\
  “ma\ \_\ \_\ er” + “\_\ \_\ ng\_\ \_ = “manger”\\
  “abra” ++ “cadabra” = “abracadabra”

\end{ex}\ \\

\begin{rem}{Lien entre Mots et Alphabet}
  $\{M \in \tn{Mots}\ |\ \tn{len}(M) = n\} = (\tn{Alphabet}\cup\{“\_”\})^n$
\end{rem}

\begin{prop}{Symétrie}
  pal est une symétrie orthogonale pour le produit scalaire usuel.
\end{prop}

\begin{dem}
  pal est linéaire pour (+) et $\tn{pal}\circ\tn{pal}=\tn{id}$
\end{dem}

\begin{prop}{Identité de Benichou-Battistelli}
  Palindromes est une demi-droite vectorielle de Mots.
\end{prop}

\begin{voir}{Pliage d'un palindrome}
  \includegraphics[width=\textwidth]{images/ppal.png}
  On voit bien comment la droite se transforme en demi-droite.
\end{voir}

\begin{dem}
  Soit $m\in\tn{Mots}$, $m$ se décompose dans Alphabet donc s'écrit en coordonnées donc c'est un vecteur.
  De plus, un mot s'écrit sur une ligne donc il est droit, et par symétrie du palindrome on peut le plier en deux au milieu, formant alors une demi-droite vectorielle.
\end{dem}

\begin{defn}{Décomposition en palindromes}
  Tout mot appartenant à Mots se décompose en la concaténation d'un certain nombre de sous-mots.
  Ces sous-mots peuvent soit appartenir à Palindromes (dans ce cas, c'est un sous-palindrome),
  soit à Mots\textbackslash Palindromes = Mots$^-$ (dans ce cas-là, c'est un sous\tutu mot).\\
  On appelle décomposition d'un mot $m$ un n-uplet $(m_i)_{1\leqslant i\leqslant n}$ de sous-mots de $m$
  vérifiant : $\displaystyle\conc_{1\leqslant i\leqslant n}m_i=m$\\
  On parle de \textbf{LA} décomposition de $m$ notée $\tn{dec}(m)$ pour désigner celle qui fait intervenir
  le plus de lettres dans les palindromes. Si deux décompositions font intervenir autant de lettres,
  on introduit un ordre lexicographique sur le rang (position de la première lettre du sous-mot dans le mot) des sous-palindromes.
  La décomposition de mot est donc unique.
  De la même manière, on appelle ced la fonction qui à partir d'une décomposition (donc n-uplet) renvoie la concaténation des sous-mots.
\end{defn}

\begin{ex}{Décompositions}
  Dans le mot “zoologique”, on a les sous-palindromes “oo” et “olo”. Or “oo\ \_\ \_” $\cap$ “\_\ olo” = “\_\ o\ \_\ \_” $\ne\emptyset$.
  On choisit la décomposition faisant intervenir “olo” car len(“olo”) = 3 > len(“oo”) = 2.\\
  Ainsi : dec(“zoologique”) = (“zo”, “olo”, “gique”)

  Dans le mot “maracas” (instrument de musique), on a les sous-palindromes “ara” et “aca”.
  Or : $\tn{“ara\ \_\ \_”}\cap\tn{“\_\ \_\ aca”}=“\tn\_\ \_\ {\tn{a}\ \_\ \_}$”. On choisit la décomposition faisant intervenir “ara” car rang(“ara”) = 2 < rang(“aca”) = 4.\\
  Ainsi : dec("maracas") = (“m”, “ara”, “cas”)
\end{ex}

\begin{defn}{pals, palt}
  On introduit les fonctions :
  $$\tn{palt} : \ \begin{aligned}
      \tn{Mots}^n                          & \rightarrow\tn{Mots}^n                          \\
      (\tn{m}_i)_{1\leqslant i\leqslant n} & \mapsto(\tn{m}_{n-i})_{1\leqslant i\leqslant n}
    \end{aligned}$$
  $$\tn{pals} : \ \begin{aligned}
      \tn{Mots}^n                          & \rightarrow\tn{Mots}^n                                \\
      (\tn{m}_i)_{1\leqslant i\leqslant n} & \mapsto(\tn{pal}(\tn{m}_i))_{1\leqslant i\leqslant n}
    \end{aligned}$$
  Alors : $$\tn{pal}=\tn{ced}\circ\tn{palt}\circ\tn{pals}\circ\tn{dec}$$
\end{defn}

\begin{rem}{Appellation et symétrie}
  Le t de palt signifie total et le s de pals signifie sous.\\
  palt et pals sont également des symétries.
\end{rem}\ \\

\begin{ex}{Utilisation de pals et palt}
  Sur le mot “maracas” :
  \begin{align*}
    \tn{pal}(\tn{“maracas”}) & =\tn{ced}\circ\tn{palt}\circ\tn{pals}\circ\tn{dec}(\tn{“maracas”})        \\
                             & =\tn{ced}\circ\tn{palt}\circ\tn{pals}(\tn{“m”},\ \tn{“ara”},\ \tn{“cas”}) \\
                             & =\tn{ced}\circ\tn{palt}(\tn{“m”},\ \tn{“ara”},\ \tn{“sac”})               \\
                             & =\tn{ced}(\tn{“sac”},\ \tn{“ara”},\ \tn{“m”})                             \\
                             & =\tn{“sacaram”}
  \end{align*}
\end{ex}

\begin{histoire}{Mattéo Leturcq\tutu Daligaux}
  \includegraphics[width=\textwidth]{images/matteo.jpg}

  Mattéo Leturcq\tutu Daligaux (2003-) est un mathématicien français né à Paris qui étudie à la prépa
  très prestigieuse Pierre de Fermat pour pouvoir en apprendre toujours plus sur les mathématiques.
  C'est dans le cadre de ses études sur la très récente théorie des mots qu'il découvre son théorème
  majeur et invente les fonctions pal, pals, palt et verlan, permettant ainsi un grand développement
  de la théorie.
\end{histoire}

\begin{ana}{Étude de la symétrie pals}
  Dans l'exemple précédent, lorsque l'on a appliqué : $$\tn{pals}\circ\tn{dec}(\tn{“maracas”})=(\tn{“m”},\ \tn{“ara”},\ \tn{“sac”})$$
  on se retrouve avec une partie palindromiale inchangée : $\tn{“\_\ ara\ \_\ \_\ \_”}$
  et une partie non-palindromiale inversée : $\tn{“m\ \_\ \_\ \_\ sac”}$.
  Si par la suite on applique palt, on brise la symétrie car on change aussi la partie palindromiale, qui devient ici $\tn{“\_\ \_\ \_\ ara\ \_”}$.
  C'est pourquoi on s'intéresse à $\tn{ced}\circ\tn{pals}\circ\tn{dec}$, on a même les éléments caractéristiques de la symétrie :

  Soit $m\in \tn{Mots}$,
  $$\tn{dec}(m)=(m_i)_{1\leqslant i\leqslant n}=(\epsilon_\tn{Pal}(m_i))_{1\leqslant i\leqslant n}+(\epsilon_{\tn{M}^-}(m_i))_{1\leqslant i\leqslant n}$$
  Avec : \begin{align*}
    \epsilon_\tn{Pal}:m_i   & \mapsto m_i\tn{ si }m_i\in\tn{Palindromes},\ “\_\ \_\ \dots\ \_”\tn{ sinon} \\
    \epsilon_{\tn{M}^-}:m_i & \mapsto m_i\tn{ si }m_i\in\tn{Mots}^-,\ “\_\ \_\ \dots\ \_”\tn{ sinon}
  \end{align*}
  Il s'agit de la décomposition de $m$ suivant la somme directe $\tn{Mots}=\tn{Palindromes}\oplus\tn{Mots}^-$, que l'on note : $m=m_\tn{Pal}+m_{\tn{M}^-}$\\
  Ainsi : \begin{align*}
    \tn{ced}\circ\tn{pals}\circ\tn{dec}(m) & =\tn{ced}\circ\tn{pals}((\epsilon_\tn{Pal}(m_i))_{1\leqslant i\leqslant n}+(\epsilon_{\tn{M}^-}(m_i))_{1\leqslant i\leqslant n})       \\
                                           & =\tn{ced}((\tn{pal}(\epsilon_\tn{Pal}(m_i)))_{1\leqslant i\leqslant n}+(\tn{pal}(\epsilon_{\tn{M}^-}(m_i)))_{1\leqslant i\leqslant n}) \\
                                           & =\tn{ced}((\epsilon_\tn{Pal}(m_i))_{1\leqslant i\leqslant n})+\tn{ced}((\tn{pal}(\epsilon_{\tn{M}^-}(m_i)))_{1\leqslant i\leqslant n}) \\
                                           & =m_\tn{Pals}+\tn{ced}\circ\tn{pals}\circ\tn{dec}(m_{\tn{M}^-})
  \end{align*}
  On en déduit le théorème suivant :
\end{ana}

\begin{thm}{Théorème du Turc d'Aligot (HP)}
  $\tn{ced}\circ\tn{pals}\circ\tn{dec}$ est une symétrie de direction $\tn{Mots}^-$ par rapport à Palindromes.
\end{thm}

\section{Démonstration de la suprématie des MP3}

\begin{heu}{Utilité de la fonction palindrome}
  Les deux premières démonstrations se concentrent sur des résultats sans trop de lien avec le 2.1.
  Elles se concentreront sur l'utilisation de la fonction pal.
  Nous allons ainsi démontrer la suprématie des MP3 par rapport aux autres filières puis utiliser la propriété de somme directe.
\end{heu}

\begin{att}{Ordre strict}
  Ne pas confondre la relation d'inclusion
  avec la relation d'ordre strict de la suprématie (<).
  Cependant, l'inclusion implique la quasi-suprématie : \\
  $\forall C1, C2 \in \Part(\EF),\ C1 \subset C2 \Rightarrow C1 \leqslant C2$
\end{att}

\setcounter{subsection}{-1}
\subsection{ECS}

\begin{thm}{Sur les ECS}
  Les ECS n'existent pas.
\end{thm}

\begin{dem}
  Soit l'évènement ECS = “Les ECS existent”.\\
  pal(“ECS”) $=$ “SCE” donc $\Prob$[pal(“ECS”)] = $\Prob$[“SCE”] = $\Prob$(SCE) \\
  Par chance, $[] : (\tn{Mots}, \tn{pal}) \rightarrow (\Omega,\complement)$ est un morphisme bijectif de groupes
  (pas démontré en cours parce que “pas le temps”) donc
  “SCE” = [Pal(“ECS”)] $= \complement(\tn{ECS})$ ainsi $\Prob(\tn{ECS}) = 1 - \Prob$[“SCE”] $= 1-\Prob(\tn{SCE})$ \\
  Or, SCE est un Système Complet d'Évènements donc $\Prob$(SCE) = 1 donc $\Prob(\tn{ECS}) = 0$ \\
  Donc les ECS n'existent pas (presque sûrement).
\end{dem}

\subsection{HK}

\begin{thm}{Sur les HK}
  Les HK sont vides, contrairement aux MPSI3 (ex : Adam $\in$ MPSI3) et donc MPSI3 > HK.
\end{thm}
\begin{dem}
  pal(“HK”) $=$ “KH” donc pal(“HK”) est KH-intégrable, ainsi
  $$\int_0^1{\tn{pal(“HK”)}d\tn{kh}}
    = \int_0^1{\tn{“KH”}d\tn{kh}}$$
  Or les hypokhâgnes (HK) (entre la 0\textsuperscript{e} et la 1\textsuperscript{ère} année)
  dérivent en khâgnes (KH) et de plus, HK ne dépend pas de KH car ils sont sur 2 cycles différents donc :
  \begin{align*}
    \tn{“KH”} = \tn{“KH”}\int_0^1{d\tn{kh}} & = \tn{“HK”}'\int_0^1{d\tn{kh}}                                 \\
                                            & = \int_0^1{\tn{“HK”}'d\tn{kh}}\ \tn{ par indépendance}         \\
                                            & =[\tn{“HK”}]_0^1 = \tn{“HK”}\ \tn{ par th. fond. de l'analyse}
  \end{align*}
  Donc “KH” $=$ “HK” càd pal(“HK”) $=$ “HK” \\
  Ainsi $\tn{“HK”}\in\tn{Palindromes}$ \\
  Il n'en reste pas moins que : “H” $\ne$ “K” donc $\tn{“HK”}\notin\tn{Palindromes}$
  $\tn{Pal}\cap\tn{Mots}^-=\{“\_”\}$, on conclut que $\tn{“HK”}=“\_”$.\\
  Ainsi les HK sont vides.
\end{dem}

\subsection{Chartes}

\begin{voir}{Chartreux}
  \includegraphics[width=\textwidth]{images/chartreux.png}
\end{voir}

\begin{thm}{Sur les Chartes}
  Les Chartes sont seuls et flemmards.
\end{thm}

\begin{dem}
  Commençons par établir que : $$\tn{ux} \times \tn{Charte} = \tn{Chartreux} = \tn{Chat (illustration)}$$
  Donc : $$\tn{pas}\times(\tn{ux} \times \tn{Charte}) = \tn{pas}\times\tn{Chat} = \tn{Pacha} = \tn{Flâneur}$$
  Ainsi $$\tn{ux}\times\tn{Charte} = \tn{pas}^{-1}\times\tn{Flâneur} = \tn{id}\times\tn{Flâneur}=\tn{Flâneur}$$
  Or, ux est l'abbréviation de uxor (mot latin signifiant époux, marié) donc : $\tn{ux}^{-1}=\tn{seul}$\\
  Donc : $\tn{Charte} = \tn{seul}\times\tn{Flâneur}$\\
  Ainsi tout Charte est seul et flemmard. Ce résultat est alors valable pour toute la classe Chartes.
\end{dem}

\begin{cor}{En termes de suprématie}
  Chartes < MPSI3.
\end{cor}

\begin{dem}
  Sur l'échelle de volonté, les MPSI3 sont aussi flemmards que les Chartes.
  Cependant, sur l'échelle de la quantité, les MPSI3 sont loin d'être seuls (ils sont même 49).
  Donc, par comparaison avec l'ordre lexicographique : MPSI3 > Chartes.
\end{dem}

\subsection{Bio}
\subsubsection{\underline{Grenouilles}}

\begin{thm}{Sur les Grenouilles}
  La classe des Grenouilles est nulle, à l'inverse de la MPSI3. Il en découle la suprématie de la MPSI3 sur les Grenouilles.
\end{thm}

\begin{histoire}{Léo Guillet}
  \includegraphics[width=174pt]{images/leo.jpg}
  Léo Guillet : Né le 17 Janvier 2002, coincheur et scientifique de renom, il fut le
  père fondateur de l'étude de l'omniprésence féminine des bios. Il prédit avec une exactitude et une
  précision troublante de nombreux résultats dans le domaine. À croire qu'il se chargeait de prélever
  ses théorèmes directement à la source.... Ce personnage est également entouré de controverses : ses
  talents à la coinche seraient dûs une capacité de triche hors norme et certains disent qu'il aurait
  arrêté ses recherches sur les bios pour se concentrer sur les ECS\dots
\end{histoire}

\begin{dem}
  $$\tn{Grenouilles}=\tn{Reinettes}=\tn{tte}\times\tn{Reines}$$
  On a $\proj_\tn{MPSI3}(\tn{Grenouilles})=\tn{tte}$ car Reine est une sous-classe de MPSI3\\
  Or : $\tn{Reines} = \tn{MPSI3}\cap\tn{Filles}$ car “MP3 tous des rois”, donc :
  \begin{align*}
    \tn{male}\times\tn{Grenouilles} & =\tn{male}\times\tn{tte}\times\tn{Reines}    \\
                                    & =\tn{malette}\times\tn{MPSI3}\cap\tn{Filles} \\
                                    & =\tn{Mattéo}\cap\tn{Filles}=\emptyset
  \end{align*}
  En effet Mattéo est porteur de la malette infiniment souvent. Ainsi il n'existe pas de Grenouille mâle (résultat assez intuitif, il sufit de les observer, c.f. parenthèse historique).\\
  Tout mâle possède les chromosomes $XY$ ce qui nous permet d'écrire que : $$XY\times\tn{Grenouilles}=\emptyset$$
  En multipliant à gauche par $X^{-1}=\checkmark$, on obtient :
  $$Y\times\tn{Grenouilles}=\checkmark\times\emptyset$$
  On simplifie par $\checkmark$ : $$_/\times\tn{Grenouilles}=\emptyset$$
  Puis par $/$ : $$\tn{Grenouilles}=0$$
  Ainsi les Grenouilles sont nulles.
\end{dem}
\subsubsection{\underline{Poireaux}}

\begin{thm}{Sur les Poireaux}
  Les Poireaux sont détestés des gens, tandis que
  les MPSI3, à l'image des trois terreurs de l'Ouest, sont incroyablement populaires
  donc MPSI3 > Poireaux.
\end{thm}

\begin{dem}
  \includegraphics[width=\textwidth]{images/sond.jpg}
\end{dem}

\subsection{PCSI}

\subsubsection{\underline{PCSI1}}

\begin{thm}{Sur les PCSI1}
  Les PCSI1 sont solitaires : ils ne s'entraident pas.\\
  Les MPSI3 unissant leurs forces et travaillant ensemble ont donc une cohésion de groupe plus forte que les PCSI1 et alors : MPSI3 > PCSI1
\end{thm}

\begin{dem}
  D'après le théorème de l'avantage concurrentiel (cf. 2.2.), la PCSI1 admet une base $\mathcal{B}=(e_i)_{1\leqslant i\leqslant n}$.\\
  Notons $M=\mathcal{M}_\mathcal{B}(\tn{PCSI1})$ la matrice de la famille des élèves de PCSI1 dans la base $\mathcal{B}$ :
  $$M=\begin{pmatrix}
      \substack{\tn{AdjaYanis}                                \\\displaystyle\overbrace{\lambda_{1,1}}} &
      \substack{\tn{AlmairacLouis}                            \\\displaystyle\overbrace{\lambda_{1,2}}} & \cdots &
      \substack{\tn{VignonTanguy}                             \\\displaystyle\overbrace{\lambda_{1,46}}} \\
      \lambda_{2,1} & \lambda_{2,2} & \cdots & \lambda_{2,46} \\
      \vdots        & \vdots        & \cdots & \vdots         \\
      \lambda_{n,1} & \lambda_{n,2} & \cdots & \lambda_{n,46}
    \end{pmatrix}$$
  Le noyau des PCSI1 est par définition leur mascotte, qu'ils n'ont plus (car on leur a volée), ainsi : $\ker M=\{0\}$\\
  Les colonnes de $M$ forment donc une famille libre, i.e. il n'y a aucune relation linéaire entre les PCSI1, ce qui signifie qu'ils ne travaillent pas ensemble.
\end{dem}

\begin{exercice}
  Démontrer que les PCSI1 sont des imbéciles heureux.
\end{exercice}

\begin{correction}
  Notons, pour chaque classe $C$ : $$\tn{fun}_C : \ \begin{aligned}
      \Car & \rightarrow C                   \\
      c    & \mapsto c\times\tn{fun}\times C
    \end{aligned}$$
  la fonction qui à un caractère associe l'amusement que les élèves de la classe $C$ retirent du caractère.\\
  On remarque que $\tn{fun}_\tn{PCSI1}=0$ lorsqu'elle est appliquée à pasDeBourriquet, donc
  $$\ker(\tn{fun}_\tn{PCSI1})=\tn{fun}(\ker \tn{PCSI1})=\{\tn{pasDeBourriquet}\}$$
  Rappelons que $\ker \tn{PCSI1} = \{\tn{Bourriquet}\}$ car il s'agit de leur mascotte, ainsi : $$\tn{fun}\{\tn{Bourriquet}\}=\{\tn{pasDeBourriquet}\}$$
  Or, nous avons volé la mascotte des PCSI1 donc $\{\tn{pasDeBourriquet}\}=0$ et $\{\tn{Bourriquet}\}=1$.\\
  Ainsi $\tn{fun}(0)=1$, autrement dit les PCSI1 s'amusent avec rien donc les PCSI1 sont des imbéciles heureux.
\end{correction}

\subsubsection{\underline{PCSI2}}

\begin{lem}{Lemme de Mallet}
  \centering $1=2$.
\end{lem}

\begin{histoire}{Grégoire Mallet}
  \includegraphics[width=\textwidth]{images/gregoire.jpg}
  Grégoire Mallet (2002-) né le 29 juillet, est un mathématicien et astronome français surtout connu
  pour avoir résolu des problèmes d'astronomie importants tels que la périodicité quadratique de la
  comète de Mallet. En expérimentateur de génie, il est aussi le fondateur de la mécanique des flamby.
  Ses quelques travaux en mathématiques et notamment son lemme, qu'il publia pour la première fois en
  2021 dans un article de \textit{Sciences Plotistes} lui valurent une renommée internationale.
\end{histoire}

\begin{dem}
  Dans le cours sur le calcul matriciel (chapitre 23), nous avons la propriété suivante :
  $$\forall A, B \in \mathcal{M}_n(\R),\ A^T + B^T = (A+B)^T$$
  Appliquons ce résultat avec $T=0$ et $n=1$, $A$ et $B$ sont alors identifiables à des réels $a$ et $b$, et on a :
  $$a^0+b^0=(a+b)^0\tn{ soit : }1+1=1\tn{ ou encore : }2=1$$
\end{dem}

\begin{cor}{Sur les PCSI2}
  MPSI3 > PCSI2.
\end{cor}

\begin{dem}
  On a établi dans le 3.4.1 que MPSI3 > PCSI1.\\
  Grâce au lemme de Mallet, on a immédiatement : PCSI1 $=$ PCSI2 donc MPSI3 > PCSI2
\end{dem}

\subsection{MPSI$\boldsymbol{X}$, $\boldsymbol{X\ne 3}$}

\subsubsection{\underline{MPSI1}}

\begin{defn}{Norme $\Vert\dashedph[x]\Vert_\tau$}
  La norme $\Vert\dashedph[x]\Vert_\tau : \Part(\EF) \rightarrow \N$ associe à une classe le nombre de chapitres réalisés dans une matière précisée préalablement.
\end{defn}

\begin{dem}
  Le nombre de chapitres à couvrir en prépa au cours des deux années est é-norme. Donc il s'agit bien d'une norme.
\end{dem}

\begin{thm}{Sur les MPSI1}
  $\tn{MPSI1}=o(\tn{MPSI3})$, ce qui entraîne directement que MPSI3 > MPSI1.
\end{thm}

\begin{dem}
  Les MPSI1 sont en retard sur le programme. En effet, ils ont cours la dernière semaine alors que Pichoff ne fait là que du hors programme aux MPSI3.
  Donc $\Vert\tn{MPSI1}\Vert_\tau < \Vert\tn{MPSI3}\Vert_\tau$.\\
  Soit $\epsilon > 0$. $$\epsilon\Vert\tn{MPSI3}\Vert_\tau=\Vert\epsilon\tn{MPSI3}\Vert_\tau$$
  On a vu que pal est une symétrie orthogonale donc une isométrie, ainsi pal ne change pas la norme :
  $$\Vert\epsilon\tn{MPSI3}\Vert_\tau=\Vert\tn{pal}(\epsilon\tn{MPSI3})\Vert_\tau$$
  On sait aussi que $\tn{pal}=\tn{ced}\circ\tn{palt}\circ\tn{pals}\circ\tn{dec}$, et la décomposition de $\epsilon$MPSI3
  est ($\epsilon$, MPSI3).
  En effet :
  \begin{align*}
    \tn{MPSI3 est un palindrome} & \Leftrightarrow \tn{MPSI3 n'est pas un lindrome} \\ &\Leftrightarrow \tn{MPSI3 n'est pas l'un des DROM}
  \end{align*}
  Et MPSI3 est un royaume (cf. Fermap) et donc non un DROM (Département et région d’outre-mer). Ce n’est donc pas l’un des DROM,
  ainsi MPSI3 est un palindrome donc pal(MPSI3) = MPSI3.
  Ensuite : $$\forall n,\ \tn{pal}(\epsilon)=\tn{pal}^n(\epsilon)\underset{n\rightarrow\infty}{\longrightarrow}3$$
  En effet, $\epsilon$ s'est tellement fait retourner par les pal qu'il en est désorienté.
  Donc pal($\epsilon$MPSI3) = MPSI33. Ainsi :
  $$\epsilon\Vert\tn{MPSI3}\Vert_\tau=\Vert\tn{MPSI33}\Vert_\tau=\Vert\tn{MPSI3}\times 11\Vert_\tau=11\Vert\tn{MPSI3}\Vert_\tau$$
  Donc : $\Vert\tn{MPSI1}\Vert_\tau < \epsilon\Vert\tn{MPSI3}\Vert_\tau$\\
  Finalement : $\tn{MPSI1} = o(\tn{MPSI3})$
\end{dem}

\begin{voir}{Fermap}
  \includegraphics[width=\textwidth]{images/fermap.jpg}
\end{voir}

\subsubsection{\underline{MPSI2}}

\begin{thm}{Sur les MPSI2}
  Les MPSI3 écrasent les MPSI2, ce qui a pour conséquence immédiate que MPSI3 > MPSI2.
\end{thm}

\begin{dem}
  Mme Goutelard a dit “Prenez exemple sur les MP3, eux au moins ils travaillent !” \\
  De plus, Mme Goutelard est contrainte de rajouter des heures sup aux MP2 le samedi afin d’espérer rattraper le fabuleux M. Lagoute.\\
  Ainsi, $\Vert\tn{MPSI3}\Vert_\tau > \Vert\tn{MPSI2}\Vert_\tau$  et Lagoute > Goutelard. Donc MPSI3 $>>$ MPSI2 (relation d'écrasement) par combinaison des inégalités.
\end{dem}

\begin{sfaire}{Méthodes pour prouver une suprématie}
  Pour démontrer la suprématie d'une Filière par rapport à une autre, nous avons utilisé :
  \begin{enumerate}
    \item la fonction pal pour modifier la signification du nom d'une filière
    \item la multiplication par des caractères (profiter de la malléabilité de $\Car$) pour altérer la forme d'une classe
    \item le recours à des rôles plotistes (cf. GFO) et à des informateurs provenant des différentes classes (cf. MPSI2 et Léo)
  \end{enumerate}
  Tandis que les méthodes 1. et 3. sont plus des stratégies circonstancielles se basant sur des
  coïncidences, la méthode 2., par l'utilisation des caractères, de par sa richesse quasiment infinie,
  peut à peu près tout démontrer pourvu que l'on prenne le temps de s'y atteler et d'avoir un peu d'imagination.
  Si vous ne savez pas où commencer un exercice car vous n'avez repéré aucun levier sur lequel faire travailler vos fonctions
  de référence, penchez-vous vers les caractères !
\end{sfaire}

\subsection{Général}

\begin{thm}{Théorème fondamental de Layton}
  $\forall C \in \Part(\EF),\ C \leqslant \tn{MPSI3}$\\
  Si $\proj_\tn{MPSI3}(C)=0$, alors $C < \tn{MPSI3}$.
\end{thm}

\begin{plusloin}{Structure topologique des filières}
  On peut également trouver des résultats topologiques plus généraux sur l’espace normé $\Fil$.
  Ainsi on démontre que $\Fil$ est connexe mais n’est pas connexe par arcs (en effet, il n’existe pas
  de chemins reliant \tn{ECS} et \tn{MPSI} restant dans $\Fil$). $\Fil$ est également un ouvert de $\Part(\EF)$ :
  il y a une grande ouverture entre les filières grâce aux SOIFs.
\end{plusloin}

\begin{heu}{Démonstration}
  L'idée va être que pour toute classe $C$ de $\Part(\EF)$, on peut “découper” la classe en des “morceaux” appartenant à chacune des filières.
  Or, on peut utiliser tous les théorèmes qu'on a précédemment démontrés pour trouver pour chaque morceau une sous-classe de MPSI3 supérieure au morceau.
  En recollant les morceaux supérieurs à chacun des morceaux de $C$, on obtient une sous-classe inférieure à MPSI3.
\end{heu}

\begin{dem}
  On a :
  \begin{align*}
    \Part(\EF)= & \Part(\tn{HK}) \oplus\Part(\tn{Chartes}) \oplus\Part(\tn{Grenouilles}) \oplus \\
                & \Part(\tn{Poireaux}) \oplus \Part(\tn{PCSI1}) \oplus\Part(\tn{PCSI2}) \oplus  \\
                & \Part(\tn{MPSI1}) \oplus\Part(\tn{MPSI2}) \oplus\Part(\tn{MPSI3})
  \end{align*}
  Ainsi toute classe $C\in\Part(\EF)$ s'écrit comme la somme directe de sous-filières de chaque filière :
  $$C=C_{\tn{MPSI3}} \uplus \biguplus_{\substack{F \in \Fil \\ F \ne \tn{MPSI3}}}C_F\ \tn{ où }\ C_{\tn{MPSI3}} \subset \tn{MPSI3}\tn{ et }C_{F}\subset F$$
  Or, d'après toutes les démonstrations précédentes du 3. :
  $$\forall F \in \Fil\backslash\{\tn{MPSI3}\},\ \forall G \in \Part(F),\ \exists D \in \Part(\tn{MPSI3}),\ G < D$$
  $D$ étant incluse dans MPSI3, on a : $D \leqslant \tn{MPSI3}$\\
  Alors par transitivité de $(<)$ : $G < \tn{MPSI3}$\\
  Donc : $$\biguplus_{\substack{F \in \Fil \\ F \ne \tn{MPSI3}}}C_F < \tn{MPSI3}$$
  Si $\proj_\tn{MPSI3}(C)=0$, i.e. $C_\tn{MPSI3}=0$, alors $C=\biguplus_{\substack{F \in \Fil \\ F \ne \tn{MPSI3}}}C_F$ donc $C < \tn{MPSI3}$\\
  Sinon, on a au moins : $C \leqslant \tn{MPSI3}$
\end{dem}

\section{Bilan}

\subsubsection{\underline{Synthèse}}

\begin{itemize}
  \item[$\rightsquigarrow$] L’algèbre fermatienne se base donc sur l’abus de l’usage des caractères. Il existe ainsi une pléthore de façons de démontrer la supériorité de la classe de MPSI3 grâce à l’usage et la manipulation des caractères. Choisissez la vôtre !
  \item[$\rightsquigarrow$] Les palindromes permettent d’offrir plus de possibilités de manipulation des caractères des mots et d’en tirer des propriétés par la suite applicables sur l’espace fermatien s’adaptant particulièrement bien à sa structure.
  \item[$\rightsquigarrow$] La suprématie d’une classe par rapport à une autre s'établit donc grâce à une manipulation habile, mais surtout à travers les caractéristiques intrinsèques aux classes comparées.
\end{itemize}

\subsubsection{\underline{Savoir-faire et Truc \& Astuce du chapitre}}

\begin{itemize}
  \item[216.] Savoir-faire - Méthodes pour prouver une suprématie
\end{itemize}

\subsubsection{\underline{Notations}}

\footnotesize
%\zspace\zspace\zspace\zspace\zspace\zspace\zspace\zspace\zspace\zspace
\begin{tabular}{p{2.5cm} p{5cm}p{5cm}p{4cm}}
  \hline \textit{Notations}     & \textit{Définitions}                                                         & \textit{Propriétés}                                           & \textit{Remarques}                                                             \\
  \hline $\EF$                  & Ensemble des élèves de Fermat                                                & $\FF = \AF \uplus \EF \uplus \PF$                             & Il s'agit des briques élémentaires de l'algèbre fermatienne                    \\
  $\Car$                        & Ensemble des caractéristiques des élèves de Fermat                           &                                                               & Un élément de $\Car$ est appelé caractère                                      \\
  $\Part(\EF)$                  & Ensemble des classes de $\EF$                                                & $\Part(\FF) = \Part(\AF) \oplus \Part(\EF) \oplus \Part(\PF)$ & $\Part(\EF)$ est un $\Car$-ev                                                  \\
  Alphabet                      & Ensemble des lettres                                                         & Alphabet = \{a, b, \dots, z\}                                 & Il s'agit des briques élémentaires de l'algèbre palindromiale                  \\
  Mots                          & Ensemble des listes de lettres de Alphabet $\cup$ \{“\_”\}                   &                                                               & Un élément de Mots est appelé un mot                                           \\
  pal                           & Retourne un mot                                                              &                                                               &                                                                                \\
  Palindromes                   & Sous-ensemble des mots invariants par pal                                    & Palindromes est une droite vectorielle de Mots                & Un élément de Palindromes est appelé palindrome                                \\
  Mots$^-$                      & Mots$^-$ = Mots \textbackslash Palindromes                                   &                                                               &                                                                                \\
  $p_c$                         & Projection sur sur $C$                                                       &                                                               & Renvoie la composante d’un élément selon son appartenance à $C$                \\
  $\equiv$                      &                                                                              & Relation d'équivalence                                        &                                                                                \\
  $<$                           & Relation de suprématie entre deux ensembles                                  &                                                               &                                                                                \\
  $\Fil$                        & Ensemble des filières                                                        & Elles paritionnent $\EF$                                      &                                                                                \\
  $\Vert\dashedph[x]\Vert_\tau$ & Norme d’un vecteur représentant le nombre de chapitre que le vecteur a finis & C'est une norme                                               & Elle permet de classer les classe en fonction du travail mathématique effectué
\end{tabular}

\subsubsection{\underline{Retour sur les problèmes}}

\begin{itemize}
  \item[159.] Les lois d'interaction des ensembles permettent dans un premier temps de gérer les élèves,
    mais la connaissance du langage à travers les mots et les palindromes notamment permet d'approfondir
    cette maîtrise des interactions fermatiennes.\\
  \item[160.] La MP3, étant composée de Rois et Reines, se situe au pinacle de la hiérarchie fermatienne.
    On a donc démontré que toutes les autres classes étaient dominées et donc assujetties par la MP3
    grâce à notre fantastique versatilité dans tous les domaines. Comme l'a dit la reine d'angleterre
    sur memep3 : “La MP3 est la meilleure classe de Fermat”.
\end{itemize}