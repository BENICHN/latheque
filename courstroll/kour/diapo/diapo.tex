\documentclass{beamer} % Version pour la projection
%\documentclass[trans]{beamer} %Version pour l'impression
\usepackage[utf8]{inputenc}
\usepackage[french]{babel}
\usepackage{beamerthemesplit}

\usetheme{Montpellier}

\useoutertheme{sidebar}
%\usecolortheme{}
\usepackage{amsfonts,amsmath,amssymb}
%\usepackage[french,boxruled,linesnumbered,fillcomment,lined]{algorithm2e}
%\usepackage{epic}
\usepackage{fancyhdr}
\usepackage{boites,boites_exemples}
\usepackage{graphicx}
%\usepackage{makeidx}
%\usepackage{soul}
%\mtcselectlanguage{francais}
%\usepackage{vector}
%\usepackage[metapost]{mfpic} 
%\usepackage{subfigure}
\usepackage[T1]{fontenc}
%\usepackage{lxfonts}
\usepackage{fouriernc}
\usepackage[scaled=0.875]{helvet}
\usepackage{courier}
\usepackage{frcursive}

\setbeamertemplate{blocks}[rounded]%
[shadow=true]

%\usetheme{Singapore}
%%\usetheme{Copenhagen}
%\usetheme{Montpellier}
%\usepackage{auto-pst-pdf}

%\usepackage{hyperref}
\usepackage{multimedia}
%\usepackage{movie15}


\usepackage{color}
\usepackage{tikz}

\usepackage{color}

\usecolortheme{beaver}

\usecolortheme{seahorse}
\usecolortheme{rose}

%PSTRICKS FILES -------------------------------------------------------------
%\usepackage{pstcol} % To use the standard "color" package with PSTricks
%\usepackage{pst-node}
%\usepackage{pst-plot}
%\usepackage{xcolor}
%\usepackage{pst-eucl}
%\usepackage{pst-tree}
%\usepackage{pstricks}

% PSTricks extension to add ]-[ arrow lines
\makeatletter
\def\pst@arrowtable{,<->,<<->>,>-<,>>-<<,(-),[-],]-[}
\def\tx@BracketOut{BracketOut }
\@namedef{psas@[}{%
/BracketOut { CLW mul add dup CLW sub 2 div /x ED mul CLW 1.75 mul sub
/y ED /z CLW 2 div def x neg y moveto x neg CLW 2 div L x CLW 2 div
L x y L stroke 0 CLW moveto } def
\psk@bracketlength \psk@tbarsize \tx@BracketOut}
\makeatother


\usepackage{listings}

\lstset{frame=tbR,
basicstyle=\small,
keywordstyle=\bf \color{black},
%identifierstyle=\underline,
commentstyle=\it \color[gray]{0.5},
stringstyle=\color{red},
showstringspaces=false,
numbers=left,
numberstyle=\tiny \bf \color{blue},
numberfirstline=true,
linewidth=10cm,xleftmargin=1cm,%aboveskip=5mm,belowskip=1 cm,
float=!h,language=Python}

\DeclareMathOperator*{\conc}{\scalerel*{++}{\sum}}
\usepackage{scalerel}
\usepackage{dashbox}

\newcommand\dashedph[1][H]{\setlength{\fboxsep}{0pt}\setlength{\dashlength}{2.2pt}\setlength{\dashdash}{1.1pt} \dbox{\phantom{#1}}}


% THEOREM Environments ---------------------------------------------------
 
\newtheorem{Theo}{Théorème}
 \renewcommand{\theTheo}{}
 \newenvironment{thm}[1]{\begin{breakbox}\vspace*{-0,4cm}\begin{Theo}\begin{upshape}\textbf{- #1} $ $\\ }%
 {\end{upshape}\end{Theo}\end{breakbox}}
 \newtheorem{Coro}{Corollaire}
 \renewcommand{\theCoro}{}
 \newenvironment{cor}[1]{\begin{Coro}\begin{upshape}\textbf{- #1} $ $\\ }%
 {\end{upshape}\end{Coro}}
 \newtheorem{Lemme}{Lemme}
 \renewcommand{\theLemme}{}
 \newenvironment{lem}[1]{\begin{Lemme}\begin{upshape}\textbf{- #1} $ $\\ }%
 {\end{upshape}\end{Lemme}}
 \newtheorem{Propos}{Proposition}
 \renewcommand{\thePropos}{}
 \newenvironment{prop}[1]{\begin{breakbox}\vspace*{-0,4cm}\begin{Propos}\begin{upshape}\textbf{- #1} $ $\\ }%
 {\end{upshape}\end{Propos}\end{breakbox}}
 \newtheorem{Defi}{Définition}
 \renewcommand{\theDefi}{}
 \newenvironment{defn}[1]{\begin{Defi}\begin{upshape} \textbf{- #1} \end{upshape}
 $ $\\}%\begin{bfseries}
 {\end{Defi}}

 \newenvironment{rem}[1]{\noindent \textbf{Remarque - #1} $ $\\}{}
 \newenvironment{ex}[1]{\noindent \textbf{Exemple - #1} $ $\\}{}
 \newenvironment{app}[1]{\noindent \textbf{Application - #1} $ $\\}{}
 \newtheorem{Demonstration}{Démonstration}
 % DÈfinition d'une structure de thÈorËme intitulÈ
 % Demonstration
 \renewcommand{\theDemonstration}{}
 % Suppression de la numÈrotation de la structure
 % Demonstration
 \newenvironment{dem}{\begin{Demonstration}\begin{upshape} $ $\\ \small}%
  {$\square$\end{upshape}\end{Demonstration}}
 \newenvironment{exercice}{\underline{\textsf{Exercice}} $ $\\ \sffamily}{}
 \newenvironment{info}{\underline{\textbf\texttt{INFORMATIQUE}} $ $\\  \ttfamily}{}
 %\newenvironment{info}{\hspace*{1cm} \begin{informatique}}{}{\end{informatique}}
 %\geometry{margin={1cm,1cm}}


 %-------------------------------------------------------------------------
 %MATH---------------------------------------------------------------------
 
  \newcommand{\Perp}{\perp\!\!\!\!\!\perp}
  \newcommand{\Reel}{\mathbf{Re}}
 \newcommand{\Ima}{\mathbf{Im}}
 \newcommand{\dd}{\textrm{d}}
 \newcommand{\Bigcupplus}{+ \hspace{-10pt}\Bigcup}
 \newcommand{\Bigcupplusdem}{+ \hspace{-9.2pt}\Bigcup}
 \newcommand{\cupplus}{\uplus}
 \newcommand{\R}{\mathbb{R}}
 \newcommand{\K}{\mathbb{K}}
 \newcommand{\C}{\mathbb{C}}
 \newcommand{\N}{\mathbb{N}}
 \newcommand{\Z}{\mathbb{Z}}
 \newcommand{\Q}{\mathbb{Q}}
 \newcommand{\U}{\mathbb{U}}
 \newcommand{\F}{\mathbb{F}}
 \newcommand{\D}{\mathbb{D}}
 \newcommand{\re}{\textrm{Re }}
 \newcommand{\im}{\textrm{Im }}
 \newcommand{\Lin}{\mathcal{L}}
 \newcommand{\Rel}{\mathcal{R}}
 \newcommand{\Pro}{\mathcal{P}}
 \newcommand{\Proq}{\mathcal{Q}}
 \newcommand{\GLin}{\mathcal{GL}}
 \newcommand{\Ce}{\mathcal{C}}
 \newcommand{\Mat}{\mathcal{M}}
 \newcommand{\tr}{\textrm{tr }}
 \newcommand{\Fon}{\mathcal{F}}
 \newcommand{\Ker}{\textrm{Ker }}
 \newcommand{\Imm}{\textrm{Im }}
 \newcommand{\rang}{\textrm{rang }}
 \newcommand{\diag}{\textrm{diag} }
 \newcommand{\Coef}{\textrm{Coef}}
 \newcommand{\Card}{\mathrm{Card }}
 \newcommand{\Tr}{\textrm{Tr }}
 \newcommand{\vect}{\mathrm{vect}}
 \newcommand{\Vect}{\mathrm{vect}}
 \newcommand{\m}{\mathcal{A}}
 \newcommand{\M}{\mathcal{M}}
 \newcommand{\Dom}{\mathcal{D}}
 \newcommand{\Base}{\mathcal{B}}
 \newcommand{\W}{\mathcal{W}}
 \newcommand{\X}{\mathcal{X}}
 \newcommand{\Esp}{\mathbf{E}}
 \newcommand{\Var}{\mathbf{V}}
 \newcommand{\Cov}{\mathbf{Cov}}
 \newcommand{\rg}{\mathrm{rg}}
 \newcommand{\summ}{\sum\limits} 
 \newcommand{\Int}{\displaystyle \int}
 \newcommand{\Sum}{\displaystyle \sum}
 \newcommand{\prodscal}[2]{\langle #1 \, , \,  #2 \rangle} %produit scalaire
 \newcommand{\prodscala}[2]{\langle #1 \, | \,  #2 \rangle} %produit scalaire
 \newcommand{\Bigcup}{\displaystyle \bigcup}
 \newcommand{\Bigcap}{\displaystyle \bigcap}
 \newcommand{\Lim}{\displaystyle \lim}
 \newcommand{\Prod}{\displaystyle \prod}
 \newcommand{\Binom}{\displaystyle \binom}
 \newcommand{\sh}{\textrm{sh}}
 \newcommand{\ch}{\textrm{ch}}
 \newcommand{\Arctan}{\textrm{Arctan }}
 \newcommand{\transpose}[1]{{\vphantom{#1}}^{\mathit t}{#1}}
 \newcommand{\ligne}{$ $ \\ $ $
 
 }
 \newcommand{\ptt}{\forall \textrm{ }}
 \newcommand{\exs}{\exists \textrm{ }}
 \newcommand{\tq}{\ | \ }
 \newcommand{\aspace}{\hspace*{0.5cm}}
 \newcommand{\bspace}{\hspace*{1cm}}
 \newcommand{\cspace}{\hspace*{1.5cm}}
 \newcommand{\zspace}{\hspace*{-0,6cm}}
 
 \newcommand{\yyspace}{\vspace*{-0,2cm}}
 \newcommand{\zzspace}{\vspace*{-0,6cm}}
 \newcommand{\partie}[1]{\textbf{\large{#1}}}
 \newcommand{\souspartie}[1]{\textbf{#1}}
 \newcommand{\oie}{[\!\![}
 \newcommand{\fie}{]\!\!]}


\title%[Pourquoi les mathématiques en PC ?\\
%Revoir les bons réflexes de démonstration]
{Leçon 117 -  Suprématie plotiste}
\logo{ \includegraphics[width=1.5cm]{images/fmtbas.jpg}}
%\author[AP]{Antoine Pichoff}
%\institute{ECE - Lycée Vial}
\date{25 juin 2021}

\begin{document}

\begin{frame}[plain]{}
\end{frame}

\frame{\begin{center}\includegraphics[width=8cm]{images/fmtmil.jpg}\end{center}\titlepage}

\setbeamertemplate{background canvas}

\tableofcontents

%{\LARGE Objectifs}

\section{\textcolor{red}{$\Rightarrow$ Définir l'algèbre Fermatienne}}

\section{\textcolor{red}{$\Rightarrow$ Étudier les fonctions palindromiales}}

\section{\textcolor{red}{$\Rightarrow$ Appliquer la caractérisation pour démontrer la suprématie}}

%{\LARGE Plan du jour}
\section{$ $}

\subsection{1. Problèmes}

\frame{\tableofcontents[currentsubsection]}

\begin{frame}
    \frametitle{Problèmes}
    \textbf{Problèmes} Mettre au point l'algèbre fermatienne\\
    \vspace{15pt}
    \onslide<2->{\textbf{Problèmes} Comment classer les classes les unes par rapport aux autres ?}
\end{frame}

\newcommand\proj{p}
\newcommand\Prob{\mathbb{P}}
\newcommand\Part{\mathcal{P}}
\newcommand\EF{\mathbb{E}_{\tn{Fer}}}
\newcommand\PF{\mathbb{P}_{\tn{Fer}}}
\newcommand\AF{\mathbb{A}_{\tn{Fer}}}
\newcommand\FF{\mathbb{F}\tn{ermat}}
\newcommand\Car{\mathbb{C}\tn{ar}}
\newcommand\Fil{\mathbb{F}\tn{il}}
\newcommand\tutu{$-\hspace{0.4pt}-$}
\newcommand\tn[1]{\textnormal{#1}}

\subsection{2. Généralités}
\subsubsection{2.1. Espace vectoriel des classes dans Fermat}

\frame{\tableofcontents[currentsubsection]}

\begin{frame}
    \frametitle{Éléments de base}
    \begin{alertblock}{Définition - $\EF$ et $\Car$}
        On appelle $\EF$ l'ensemble des élèves étudiant à la prépa du lycée Fermat.
        On notera $\Car$ l'ensemble des caractéristiques que peut avoir un étudiant de $\EF$.
        Un élément de $\Part(\EF)$ est appelé classe et un élément de $\Car$ est appelé caractère.
        Par convention, les classes sont écrites en capitales et les caractères en minuscules.
        
        On peut additionner les éléments de $\Car$ et de $\Part(\EF)$ (respectivement entre eux)
        selon la loi $\cup$ (d'inverse $\cap$) et multiplier les éléments de $\Car$ par les éléments de $\Part(\EF)$ ou de $\Car$ selon la loi $\times$.
    \end{alertblock}
    
    \onslide<2->{\textbf{Exemple} Pour bien comprendre}
    
    \onslide<3->{\textbf{Remarque} Nomenclature}
    
    \onslide<4->{\textbf{Remarque} Bruni}
\end{frame}

\begin{frame}
    \frametitle{Ensemble des classes}
    \begin{alertblock}{Proposition - Structure de $\Part(\EF)$}
        $(\Part(\EF), \cup, \times)$ est un $\Car$-espace vectoriel.
    \end{alertblock}
    
    \onslide<2->{\textbf{Exercice} Faire la démonstration.}
\end{frame}

\begin{frame}
    \frametitle{Relations}
    \begin{alertblock}{Définition - Module et relation ($\equiv$)}
        Pour une classe donnée $C$, on utilisera la notation $\proj_C : \Part(\EF) \rightarrow \Car$
        pour désigner le caractère de la composante sur $C$ des éléments de $\Part(\EF)$.
        
        On notera aussi ($\equiv$) la relation telle que : \\
        $\forall C1, C2 \in \Part(\EF),\ C1 \equiv C2 \Leftrightarrow \proj_{C1} =  \proj_{C2}$
    \end{alertblock}
    
    \onslide<2->{\begin{alertblock}{Proposition - Nature de la relation ($\equiv$)}
            La relation ($\equiv$) est une relation d'équivalence.
        \end{alertblock}}
\end{frame}

\begin{frame}
    \frametitle{Relations}
    \textbf{Remarque} Inclusion\\
    \vspace{15pt}
    \onslide<2->{\textbf{Attention} Ordre partiel}
    
    \onslide<3->{\begin{alertblock}{Définition - Sous-classe}
            Une partie d'une classe $C$ est appelée sous-classe de $C$.
        \end{alertblock}}
    
    \onslide<4->{\begin{alertblock}{Proposition - Sous-espace vectoriel}
            Pour toute classe $C$ dans $\Part(\EF)$, l'ensemble des sous-classes de $C$
            est un s.e.v. de $\Part(\EF)$.
        \end{alertblock}}
    
    \onslide<5->{\textbf{Démonstration}}
\end{frame}

\begin{frame}
    \frametitle{Décomposition de $\EF$}
    \begin{alertblock}{Proposition - Somme directe}
        $\EF =  \tn{HK} \uplus \tn{Chartes} \uplus \tn{Bio} \uplus \tn{PCSI} \uplus \tn{MPSI}$
        
        Ainsi : $\Part(\EF) = \Part\tn{(HK)} \oplus \Part\tn{(Chartes)} \oplus \Part\tn{(Bio)} \oplus \Part\tn{(PCSI)} \oplus \Part\tn{(MPSI)}$ \\
        où : $\Part\tn{(Bio)} = \Part\tn{(Grenouilles)} \oplus \Part\tn{(Poireaux)}$,\\
        $\Part\tn{(PCSI)} = \Part\tn{(PCSI1)} \oplus \Part\tn{(PCSI2)}$ et\\
        $\Part\tn{(MPSI)} = \Part\tn{(MPSI1)} \oplus \Part\tn{(MPSI2)} \oplus \Part\tn{(MPSI3)}$
    \end{alertblock}
\end{frame}

\subsubsection{2.2. Base de l'espace des classes}

\frame{\tableofcontents[currentsubsection]}

\begin{frame}
    \frametitle{Base de l'espace des classes}
    \begin{alertblock}{Proposition - Dimension finie (Théorème de l'avantage concurrentiel)}
        Pour toute filière $F$ de $\Fil$, l'espace $\Part(F)$ possède une base canonique, également appelée salle de base.
    \end{alertblock}
    
    \onslide<2->{\textbf{Démonstration}}
\end{frame}

\subsubsection{2.3. Palindromes}

\frame{\tableofcontents[currentsubsection]}

\begin{frame}
    \frametitle{Palindromes}
    Pour démontrer la supériorité, il va falloir rajouter des outils
    extérieurs. Ici nous allons parler de la fonction pal.  
\end{frame}

\begin{frame}
    \frametitle{Définitions}
    \begin{alertblock}{Définition - Lettre, mot, Mot, pal et palindrome}
        Soit $n \geqslant 0$. On appelle lettre un élément de l'alphabet ou le caractère vide “\_”,
        mot un $n$-uplet de lettres et Mots l'ensemble des mots.
        On introduit la fonction :
        $$\tn{pal} : \ \begin{aligned}
                (\tn{Alphabet}\cup\{“\_”\})^n             & \rightarrow(\tn{Alphabet}\cup\{“\_”\})^n             \\
                (\tn{lettre}_i)_{1\leqslant i\leqslant n} & \mapsto(\tn{lettre}_{n-i})_{1\leqslant i\leqslant n}
            \end{aligned}$$
        On appelle palindromes les mots invariants par pal.
        
        L'ensemble des palindromes est appelé Palindromes.
        
        Sur Mots sont définies deux lois internes :
        \begin{itemize}
            \item $(++)$ est la concaténation de deux mots.
            \item $(+)$ est la superposition de deux mots lorsque deux caractères non vides ne se chevauchent pas.
        \end{itemize}
    \end{alertblock}
\end{frame}

\begin{frame}
    \frametitle{Définitions}
    \textbf{Exemple} Palindromes\\
    \vspace{15pt}
    \onslide<2->{\textbf{Remarque} Lien entre Mots et Alphabet}
\end{frame}

\begin{frame}
    \frametitle{Symétries}
    \begin{alertblock}{Proposition - Symétrie}
        pal est une symétrie orthogonale pour le produit scalaire usuel.
    \end{alertblock}
    
    \onslide<2->{\textbf{Démonstration}}
    
    \vspace{20pt}
    
    \onslide<3->{\begin{alertblock}{Proposition - Identité de Benichou-Battistelli}
            Palindromes est une demi-droite vectorielle de Mots.
        \end{alertblock}}
    
    \onslide<4->{\textbf{Démonstration}}
\end{frame}

\begin{frame}
    \frametitle{Palindromes}
    \begin{alertblock}{Définition - Décomposition en palindromes}
        Tout mot appartenant à Mots se décompose en l'union d'un certain nombre de sous-mots.
        Ces sous-mots peuvent soit appartenir à Palindromes (dans ce cas, c'est un sous-palindrome),
        soit à Mots\textbackslash Palindromes = Mots$^-$ (dans ce cas-là, c'est un sous\tutu mot).\\
        On appelle décomposition d'un mot $m$ un n-uplet $(m_i)_{1\leqslant i\leqslant n}$ de sous-mots de $m$
        vérifiant : $\displaystyle\conc_{1\leqslant i\leqslant n}m_i=m$\\
        On parle de \textbf{LA} décomposition de $m$ notée $\tn{dec}(m)$ pour désigner celle qui fait intervenir
        le plus de lettres dans les palindromes. Si deux décompositions font intervenir autant de lettres,
        on introduit un ordre lexicographique sur le rang (position de la première lettre du sous-mot dans le mot) des sous-palindromes.
        La décomposition de mot est donc unique.
        De la même manière, on appelle ced la fonction qui à partir d'une décomposition (donc n-uplet) renvoie la concaténation des sous-mots.
    \end{alertblock}
\end{frame}

\begin{frame}
    \frametitle{Palindromes}
    \textbf{Exemple} Décompositions
    \onslide<2->{\begin{alertblock}{Définition - pals, palt}
            On introduit les fonctions :
            $$\tn{palt} : \ \begin{aligned}
                    \tn{Mots}^n                          & \rightarrow\tn{Mots}^n                          \\
                    (\tn{m}_i)_{1\leqslant i\leqslant n} & \mapsto(\tn{m}_{n-i})_{1\leqslant i\leqslant n}
                \end{aligned}$$
            $$\tn{pals} : \ \begin{aligned}
                    \tn{Mots}^n                          & \rightarrow\tn{Mots}^n                                \\
                    (\tn{m}_i)_{1\leqslant i\leqslant n} & \mapsto(\tn{pal}(\tn{m}_i))_{1\leqslant i\leqslant n}
                \end{aligned}$$
            Alors : $$\tn{pal}=\tn{ced}\circ\tn{palt}\circ\tn{pals}\circ\tn{dec}$$
        \end{alertblock}}
    
    \onslide<3->{\textbf{Remarque} Appellation et symétrie}
    
    \onslide<4->{\textbf{Exemple} Utilisation de pals et palt}
\end{frame}

\begin{frame}
    \frametitle{Palindromes}
    \textbf{Analyse} Étude de la symétrie de pals
    
    \onslide<2->{\begin{alertblock}{Théorème - Théorème du Turc d'Aligot (HP)}
            $\tn{ced}\circ\tn{pals}\circ\tn{dec}$ est une symétrie de direction $\tn{Mots}^-$ par rapport à Palindromes.
        \end{alertblock}}
\end{frame}

\subsection{3. Démonstration de la suprématie des MP3}

\frame{\tableofcontents[currentsubsection]}

\begin{frame}
    \frametitle{Démonstration de la suprématie des MP3}
    \begin{block}{Heuristique - Utilité de la fonction palindrome}
        Les deux premières démonstrations se concentrent sur des résultats sans trop de lien avec le 2.1.
        Elles se concentreront sur l'utilisation de la fonction pal.
        Nous allons ainsi démontrer la suprématie des MP3 par rapport aux autres filières puis utiliser la propriété de somme directe.
    \end{block}
    
    \onslide<2->{\textbf{Attention} Ordre strict}
\end{frame}

\subsubsection{3.0. ECS}

\frame{\tableofcontents[currentsubsection]}

\begin{frame}
    \frametitle{ECS}
    \begin{alertblock}{Théorème - Sur les ECS}
        Les ECS n'existent pas.
    \end{alertblock}
    
    \onslide<2->{\textbf{Démonstration}}
\end{frame}

\subsubsection{3.1. HK}

\frame{\tableofcontents[currentsubsection]}

\begin{frame}
    \frametitle{HK}
    \begin{alertblock}{Théorème - Sur les HK}
        Les HK sont vides, contrairement aux MPSI3 (ex : Adam $\in$ MPSI3) et donc MPSI3 > HK.
    \end{alertblock}
    
    \onslide<2->{\textbf{Démonstration}}
\end{frame}

\subsubsection{3.2. Chartes}

\frame{\tableofcontents[currentsubsection]}

\begin{frame}
    \frametitle{Chartes}
    \begin{alertblock}{Théorème - Sur les Chartes}
        Les Chartes sont seuls et flemmards.
    \end{alertblock}
    
    \onslide<2->{\textbf{Démonstration}}
    
    \vspace{20pt}
    
    \onslide<3->{\begin{alertblock}{Corollaire - En termes de suprématie}
            Chartes < MPSI3.
        \end{alertblock}}
    
    \onslide<4->{\textbf{Démonstration}}
\end{frame}

\subsubsection{3.3. Bio}

\frame{\tableofcontents[currentsubsection]}

\begin{frame}
    \frametitle{Grenouilles}
    \begin{alertblock}{Théorème - Sur les Grenouilles}
        La classe des Grenouilles est nulle, à l'inverse de la MPSI3. Il en découle la suprématie de la MPSI3 sur les Grenouilles.
    \end{alertblock}
    
    \onslide<2->{\textbf{Démonstration}}
\end{frame}

\begin{frame}
    \frametitle{Poireaux}
    \begin{alertblock}{Théorème - Sur les Poireaux}
        Les Poireaux sont détestés des gens, tandis que
        les MPSI3, à l'image des trois terreurs de l'Ouest, sont incroyablement populaires
        donc MPSI3 > Poireaux.
    \end{alertblock}
    
    \onslide<2->{\textbf{Démonstration}}
\end{frame}

\subsubsection{3.4. PCSI}

\frame{\tableofcontents[currentsubsection]}

\begin{frame}
    \frametitle{PCSI1}
    \begin{alertblock}{Théorème - Sur les PCSI1}
        Les PCSI1 sont solitaires : ils ne s'entraident pas.\\
        Les MPSI3 unissant leurs forces et travaillant ensemble ont donc une cohésion de groupe plus forte que les PCSI1 et alors : MPSI3 > PCSI1
    \end{alertblock}
    
    \onslide<2->{\textbf{Démonstration}}
    
    \onslide<3->{\textbf{Exercice} Démontrer que les PCSI1 sont des imbéciles heureux.}
\end{frame}

\begin{frame}
    \frametitle{PCSI2}
    \begin{alertblock}{Lemme - Lemme de Mallet}
        \centering $1=2$.
    \end{alertblock}
    
    \onslide<2->{\textbf{Démonstration}}
    
    \vspace{20pt}
    
    \onslide<3->{\begin{alertblock}{Corollaire - Sur les PCSI2}
            MPSI3 > PCSI2.
        \end{alertblock}}
    
    \onslide<4->{\textbf{Démonstration}}
\end{frame}

\subsubsection{3.5. MPSI$X$, $X\ne 3$}

\frame{\tableofcontents[currentsubsection]}

\begin{frame}
    \frametitle{Norme}
    \begin{alertblock}{Définition - Norme $\Vert\dashedph[x]\Vert_\tau$}
        La norme $\Vert\dashedph[x]\Vert_\tau : \Part(\EF) \rightarrow \N$ associe à une classe le nombre de chapitres réalisés dans une matière précisée préalablement.
    \end{alertblock}
    
    \onslide<2->{\textbf{Démonstration}}
\end{frame}

\begin{frame}
    \frametitle{MPSI1}
    \begin{alertblock}{Théorème - Sur les MPSI1}
        $\tn{MPSI1}=o(\tn{MPSI3})$, ce qui entraîne directement que MPSI3 > MPSI1.
    \end{alertblock}
    
    \onslide<2->{\textbf{Démonstration}}
\end{frame}

\begin{frame}
    \frametitle{MPSI2}
    \begin{alertblock}{Théorème - Sur les MPSI2}
        Les MPSI3 écrasent les MPSI2, ce qui a pour conséquence immédiate que MPSI3 > MPSI2.
    \end{alertblock}
    
    \onslide<2->{\textbf{Démonstration}}
\end{frame}

\begin{frame}
    \frametitle{Méthode}
    \begin{block}{Savoir faire - Méthodes pour prouver une suprématie}
        Pour démontrer la suprématie d'une Filière par rapport à une autre, nous avons utilisé :
        \begin{enumerate}
            \item la fonction pal pour modifier la signification du nom d'une filière
            \item la multiplication par des caractères (profiter de la malléabilité de $\Car$) pour altérer la forme d'une classe
            \item le recours à des rôles plotistes (cf. GFO) et à des informateurs provenant des différentes classes (cf. MPSI2 et Léo)
        \end{enumerate}
        Tandis que les méthodes 1. et 3. sont plus des stratégies circonstancielles se basant sur des
        coïncidences, la méthode 2., par l'utilisation des caractères, de par sa richesse quasiment infinie,
        peut à peu près tout démontrer pourvu que l'on prenne le temps de s'y atteler et d'avoir un peu d'imagination.
        Si vous ne savez pas où commencer un exercice car vous n'avez repéré aucun levier sur lequel faire travailler vos fonctions
        de référence, penchez-vous vers les caractères !
    \end{block}
\end{frame}

\subsubsection{3.6. Général}

\frame{\tableofcontents[currentsubsection]}

\begin{frame}
    \frametitle{Généralisation}
    \begin{alertblock}{Théorème - Théorème fondamental de Layton}
        $\forall C \in \Part(\EF),\ C \leqslant \tn{MPSI3}$\\
        Si $\proj_{\tn{MPSI3}}(C)=0$, alors $C < \tn{MPSI3}$.
    \end{alertblock}
    
    \onslide<2->{\begin{block}{Heuristique - Démonstration}
            L'idée va être que pour toute classe $C$ de $\Part(\EF)$, on peut “découper” la classe en des “morceaux” appartenant à chacune des filières.
            Or, on peut utiliser tous les théorèmes qu'on a précédemment démontrés pour trouver pour chaque morceau une sous-classe de MPSI3 supérieure au morceau.
            En recollant les morceaux supérieurs à chacun des morceaux de $C$, on obtient une sous-classe inférieure à MPSI3.
        \end{block}}
    
    \onslide<3->{\textbf{Démonstration}}
\end{frame}

\section{$ $}

\begin{frame}
    \frametitle{Conclusion}
    \onslide<1->{\textbf{Objectifs}\\
        \textcolor{red}{$\Rightarrow$ Définir l'algèbre Fermatienne} \\
        \textcolor{red}{$\Rightarrow$ Étudier les fonctions palindromiales} \\ 
        \textcolor{red}{$\Rightarrow$ Appliquer la caractérisation pour démontrer la suprématie} \\ }
    
    \onslide<2->{\textbf{Pour le prochain cours}
        \begin{itemize}
            \item Exercice n\textdegree 777
        \end{itemize}}
\end{frame}

\end{document}

