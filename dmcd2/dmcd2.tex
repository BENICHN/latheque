\documentclass[a4paper]{article}

\usepackage[utf8]{inputenc}
\usepackage[T1]{fontenc}
\usepackage[french]{babel}
\usepackage[top=1cm, left=1cm, bottom=1cm, right=1cm]{geometry}
\usepackage{setspace}
\usepackage{soul}
\usepackage{ulem}
\usepackage{color}
\usepackage{xcolor}
\usepackage{listings}
\usepackage{upquote}
\usepackage{url}
\usepackage{graphicx}
\usepackage{wrapfig}
\usepackage{float}
\usepackage{multirow}
\usepackage{array}
\usepackage{colortbl}
\usepackage{amsmath}
\usepackage{amssymb}
\usepackage{mathrsfs}
\usepackage{mathtools}
\usepackage{amsthm}
\usepackage{makeidx}
\usepackage{empheq}
\usepackage{pgf,tikz}
\usepackage{xargs}
\usepackage{subcaption}
\usepackage{tabularx}
\usepackage{hhline}
\usepackage{enumitem}
\usepackage{fancyhdr}
\usepackage{titling}
\usepackage{dashbox}
\usepackage{eqparbox}
\usetikzlibrary{arrows, arrows.meta, bending, calc, backgrounds, patterns}
\usetikzlibrary{decorations.markings}

\newcommand\alignsymbols[2][c]{\mathrel{\eqmakebox[S][#1]{$#2$}}{}}
\input{../templates/styles/gfsartemisia.tex}

\author{BENICHOU Nathan}
\title{DM2 de calcul différentiel}

\renewcommand{\theenumi}{\textbf{\arabic{enumi}.}}
\renewcommand{\theenumii}{\textbf{\alph{enumii}.}}
\renewcommand{\theenumiii}{\textbf{\roman{enumiii}.}}

\renewcommand{\labelenumi}{\theenumi}
\renewcommand{\labelenumii}{\theenumii}
\renewcommand{\labelenumiii}{\theenumiii}
\geometry{top=2.25cm, left=1cm, bottom=2.25cm, right=1cm}

\pagestyle{fancy}
\renewcommand{\footrulewidth}{0pt}
\fancyhf{}
\lhead{\textsc{\theauthor}}
\rhead{\thetitle}
\definecolor{alizarin}{rgb}{0.905882353,0.298039216,0.235294118}
\definecolor{nephritis}{rgb}{0.152941176,0.682352941,0.376470588}
\definecolor{belizehole}{rgb}{0.160784314,0.501960784,0.725490196}
\definecolor{peterriver}{rgb}{0.203921569,0.596078431,0.858823529}

\definecolor{officeblueint}{rgb}{0.51372549,0.745098039,0.925490196}
\definecolor{officeblueext}{rgb}{0,0.388235294,0.694117647}
\definecolor{officegreenint}{rgb}{0.631372549,0.866666667,0.666666667}
\definecolor{officegreenext}{rgb}{0.188235294,0.564705882,0.282352941}
\definecolor{officeredint}{rgb}{1,0.568627451,0.596078431}
\definecolor{officeredext}{rgb}{0.831372549,0.137254902,0.078431373}
\definecolor{officeyellowint}{rgb}{0.97254902,0.858823529,0.560784314}
\definecolor{officeyellowext}{rgb}{0.870588235,0.423529412,0}
\definecolor{officepurpleint}{rgb}{0.831372549,0.57254902,0.847058824}
\definecolor{officepurpleext}{rgb}{0.62745098,0.294117647,0.658823529}
\definecolor{officegrayint}{rgb}{0.784313725,0.776470588,0.768627451}
\definecolor{officegrayext}{rgb}{0.474509804,0.466666667,0.454901961}
\definecolor{officeblackint}{rgb}{0.784313725,0.776470588,0.768627451}
\definecolor{officeblackext}{rgb}{0.22745098,0.22745098,0.219607843}
\newcommand\dashedph[1][H]{\setlength{\fboxsep}{0pt}\setlength{\dashlength}{2.2pt}\setlength{\dashdash}{1.1pt} \dbox{\phantom{#1}}}

\newcommand\indep{\perp \!\!\! \perp}
\newcommand\Esp[1]{\mathbb{E}(#1)}
\newcommand\Prob[1]{\mathbb{P}(#1)}
\newcommand\Ber[1]{\mathcal{B}(#1)}
\newcommand\N{\mathbb{N}}
\newcommand\Z{\mathbb{Z}}
\newcommand\R{\mathbb{R}}
\newcommand\Q{\mathbb{Q}}
\newcommand\M{\mathscr{M}}
\newcommand\norm[1]{\left\lVert#1\right\rVert}

\newtheorem*{prop}{Propriété}
\newtheorem*{propn}{Proposition}
\newtheorem*{lem}{Lemme}
\newtheorem*{crl}{Corollaire}
\newtheorem*{thm}{Théorème}
\newtheorem*{defn}{Définition}

\begin{document} %------------------------------------------------------------------------------------------------------------------------------------------------- DOC
  \begin{enumerate}
    \item $\Q$ est dense dans $\R$ donc, pour $\epsilon > 0$, il existe $q\in\Q$ tel que $\left|q-\sqrt{2}\right|\leqslant\epsilon$. Ainsi $d(\sqrt{2},\Q)=\underset{q\in\Q}{\inf}\left|q-\sqrt{2}\right|=0$
    \item Soit $y\in E$, notons $d=d(y,A)$.\\
    $d=\underset{x\in A}{\inf}\norm{x-y}$ donc il existe une suite $(x_n)\in A^\N$ telle que $\norm{x_n-y}\longrightarrow d$.\\
    À partir d'un rang $n_0$, on a : $\norm{x_n-y}\leqslant 2d$ donc $x_n \in\overline{\mathcal{B}}(y,2d)\cap \overline{A}$.\\
    Par compacité, il existe une extraction $\varphi$ telle que
    $x_{\varphi(n)}\longrightarrow x\in \overline{\mathcal{B}}(y,2d)\cap \overline{A}$.
    De plus, pour $n\geqslant 0$, on a : $$\norm{y-x_{\varphi(n)}}-\norm{x_{\varphi(n)}-x}\leqslant\norm{x-y}\leqslant\norm{y-x_{\varphi(n)}}+\norm{x_{\varphi(n)}-x}$$
    Par passage à la limite, puisque $\norm{x_{\varphi(n)}-y}\longrightarrow d$ et $\norm{x_{\varphi(n)}-x}\longrightarrow 0$, on a donc : $\norm{x-y}=d$.
    \item Soient $x,y\in E$. \begin{enumerate}
        \item Pour $a\in A$, $d(x,A)\leqslant \norm{x-a} \leqslant \norm{x-y} + \norm{y-a}$
        \item $d(x,A)-\norm{x-y}$ est un minorant de $\norm{y-a}$ pour $a\in A$, donc est inférieur à $d(y,A)$. D'où : $d(x,A)\leqslant\norm{x-y}+d(y,A)$.
        \item On a donc, pour $x,y\in E$, $$d(x,A)-d(y,A)\leqslant\norm{x-y}\text{\ \ \ \ \ \ \ et\ \ \ \ \ \ \ }d(y,A)-d(x,A)\leqslant\norm{y-x}=\norm{x-y}$$ soit encore : $$\left|d(x,A)-d(y,A)\right|\leqslant\norm{x-y}$$
        C'est à dire que $x\longmapsto d(x,A)$ est 1-lipschitzienne.
    \end{enumerate}
    \item Soient $A$ un fermé de $E$ et $B$ un compact de $E$ vérifiant $A\cap B=\emptyset$. Notons $\delta=\underset{\substack{a\in A \\ b\in B}}{\inf}\norm{a-b}=\underset{b\in B}{\inf}\,d(b,A)$.\\
    Il existe une suite $(a_n,b_n)\in (A\times B)^\N$ telle que $\norm{b_n-a_n}\longrightarrow\delta$. Par compacité, il existe une extraction $\varphi$ telle que $b_{\varphi(n)}\longrightarrow b\in B$. \\
    Comme en (2.), on a, pour $n\geqslant 0$ : $$\norm{b-b_{\varphi(n)}}-\norm{b_{\varphi(n)}-a_{\varphi(n)}}\leqslant\norm{b-a_{\varphi(n)}}\leqslant\norm{b-b_{\varphi(n)}}+\norm{b_{\varphi(n)}-a_{\varphi(n)}}$$
    et on déduit que : $\norm{b-a_{\varphi(n)}}\longrightarrow\delta$, donc $d(b,A)=\delta$. Or, $A$ est fermé donc il existe $a\in A$ tel que $\norm{b-a}=\delta$. \\
    Sachant que $A\cap B = \emptyset$, on ne peut pas avoir $\delta=0$ (auquel cas on aurait $a=b\in A\cap B$) donc $\delta>0$.
    \item Avec : $$A=\left\{\left(x,\dfrac{1}{x^2}\right) \bigg|\, x<0\right\}\text{\ \ \ \ \ \ \ et\ \ \ \ \ \ \ }B=\left\{\left(x,\dfrac{1}{x^2}\right) \bigg|\, x>0\right\}$$
    $A$ et $B$ sont fermés et disjoints, et on a : $\underset{\substack{a\in A \\ b\in B}}{\inf}\norm{a-b}=0$. En effet, posons : 
    $$(a_n)=\left(-\dfrac{1}{n},n^2\right)\in A^\N\text{\ \ \ \ \ \ \ et\ \ \ \ \ \ \ }(b_n)=\left(\dfrac{1}{n},n^2\right)\in B^\N$$
    Alors : $\norm{a_n-b_n}_2=\dfrac{2}{n}\longrightarrow 0$
  \end{enumerate}
\end{document}